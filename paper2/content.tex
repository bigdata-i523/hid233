\title{Big Data Applications in Virtual Assistants}

\author{Jiaan Wang}
\affiliation{%
  \institution{Indiana University Bloomington}
  \streetaddress{3209 E 10 St}
  \city{Bloomington} 
  \state{IN} 
  \postcode{47408}
}
\email{jervwang@indiana.edu}

\begin{abstract}

    In the age of big data, artificial intelligence and speech recognition 
    techniques have been widely used in numerous big data technologies and 
    applications. Among those are virtual assistants which could potentially 
    lead to the future evolution of big data. We list various virtual assistants 
    currently in the industry developed by giants such as Google, Microsoft, 
    Amazon and Apple. We then follow up by discussing some future development 
    of virtual assistants.
    
\end{abstract}

\keywords{i523, HID233, Big data, Virtual Assistants, Artificial intelligence}

\maketitle

\section{Introduction}

Since the early 2000s, big data has been a popular word among the tech industries. It is a term that describes data sets which are so huge that normal data management techniques are not enough to process them. Nowadays, big data usually refers to data science analytics methods such as predictive modeling, machine learning, data mining, speech recognition and so on \cite{Tal2015internet}. Big data is typically described by the 4Vs, that is Volume, Velocity, Variety and veracity. Volume stands for the size of the data, velocity refers to the data processing speed, variety means different types of data and finally veracity represents the quality of the data. 

As the name suggests, big data is big. The size of the data could go up to petabytes or even exabytes which is 1000 petabytes. Previously, to process this amount of data is essentially impossible. However, as the year goes by, the ability to analyze big data has increased significantly thanks to improved mathematical algorithms and powerful computers \cite{Tal2015internet}. 

In the era of big data, the opportunities are immense. Big data not only will help all kinds of companies and organizations to make better services and decisions but also will aid us in everyday tasks such as writing emails, sending texts, shopping online, listening to music, watching TV, traveling around the globe, managing schedules, and so on \cite{Tal2015internet}. This led to to the birth of virtual assistants which are software that can perform various tasks described above for individuals. The major technologies behind virtual assistants are artificial intelligence and speech recognition which would not be possible without the use of big data \cite{Baron2017assistants}. At present, more and more companies and organizations are beginning to employ the use of virtual assistants which utilize artificial intelligence. As a result, along with the help of machine learning algorithms, more and more applications, technologies and systems will be able to obtain information on their own such as user behaviors to predict and provide useful suggestions to individuals with more personalized experiences \cite{Hard2014applications}.

Virtual assistants work usually via three different methods: by texts, or voices or sometimes images. A virtual assistant may work via more than one method such as Goggle Assistant which can understand texts and voices. After users input their commands either via texts or voices, the virtual assistants process those information received using a method called natural language processing which then translates user inputs into executable commands they can interpret. 

\section{Current applications in the industry}

In this new era of virtual assistants, four major tech companies come into play: Google with Google assistant and Google Now, Apple with Siri, Microsoft with Cortana and Amazon with Alexa. All these virtual assistants have been successful in the market but the technologies behind them are still immature. The four giants have been competing with each other in this area for years and each of them wants to bring their own unique virtual assistant to the market. 

Google has a variety of services such as its famous search engine Google, Gmail, Google maps, Google drive, etc. and signing up for a Google account gives you the ability to use all these services any time from any internet devices such as phones, tablets and laptops. Everything you do on these services is registered and kept in a personal cloud on Google's servers. With these information, Google assistant can learn your behaviours and styles in order to provide you with useful suggestions and recommendations before you even know it \cite{Tal2015internet}.

Compared to other companies, Google has a competitive edge in machine learning because Google has access to vast amount of information and resources from its popular search engine and cloud services such as Gmail, Google drive, Google docs, etc. In addition, Google also is connected with more than 1.5 billion smart-phone users on Android platform. These people are more likely to use Google's new virtual assistant systems in the future to manage their everyday tasks \cite{Tal2015internet}.

During its original release with the IPhone 4S back in 2011, Siri was a big surprise to the general public and it became a huge success as an early virtual assistant in the field. However, the initial release of Siri also received some negative reviews for its lack of information to give directions to nearby places as well as for its bad speech recognition to understand some certain English accents. Since then, Apple has been making progressive improvements on Siri in order to overcome these disadvantages it had before \cite{Waters2015life}. With information and data collected from Apple users all over the world, either their IPhone or laptops or desktops, Apple can provide their users with a more advanced Siri to help their everyday lives \cite{Tal2015internet}.

In 2015, along with Microsoft's release of its newest operating system Windows 10 was the introduction of Cortana, the Microsoft version of a virtual assistant, finally entering the competition with Google and Apple. Furthermore, Microsoft provided the option for Windows users to download a Cortana app on their smart-phones so that they can share information across platforms (phones and computers) to manage their tasks on the go \cite{Tal2015internet}. 

The difference between Microsoft and its competitors is that Microsoft has more experience in business software such as Microsoft Power BI. As a result, Cortana is integrated with various Microsoft apps like Skype and services to aid users in making better business decisions with ease. Cortana uses reminders to schedule your meetings, create to-do lists and so on. For example, you can create a reminder like, {\em Schedule a meeting with Tom at 10am}, in an app called Smart Sticky Notes either by telling Cortana with voice command or typing it in. Then Cortana will add that reminder to your list while keeping track of it for you. In addition, Microsoft is trying to add features to Cortana such as synchronizing reminders across different devices. This could potentially lead to the expansion of Cortana to create reminders and to-do lists with data from Office 365 \cite{Marvin2017them}.

As Google Assistant, Siri and Cortana dominated the virtual assistant market and continued to get smarter, Amazon - the world's largest Internet-based retailer, wanted a piece of the action as well. In 2014, Amazon entered the competition with its Amazon Alexa which was capable of voice integration, providing real time traffic, weather and news information, playing music, setting alarms and timers as well as making to-do-lists and reminders. Alexa can also be used as a smart home system where it can control smart home devices such as lights and thermostats \cite{Marvin2017them}. 

What separates Amazon with its competitors is that in 2015 Amazon released the Alexa Skills Kit which allowed designers and developers to build their own apps and skills through the technologies behind Alexa and integrate them in any Alexa-enabled devices. The Alexa Skills Kit received highly critical claims and subsequently, companies and organizations now have a wide range of apps and skills to choose from for their own business needs. Google and Microsoft soon pursued Amazon in its path as well with the release of Google and Cortana Skills \cite{Marvin2017them}.

Eventually, in the future, people will only want one virtual assistant which is available anywhere you go and can do various tasks you request. However, with the current trend in the industry, we will mostly likely to see different virtual assistants specialize in their own unique way. For example, Google Assistant will excel in giving driving directions and gathering information while Microsoft Cortana will focus on providing people with a powerful and enjoyable gaming experience. Alexa will specialize in bringing well personalized recommendations for shopping \cite{Baron2017assistants}.

\section{The future of virtual assistants}

The possibility for the future of virtual assistants are vast. For shopping, the next generation virtual assistants should be able to provide recommendations with a focus on direct customer interactions and also use locations to create a smooth and more localized shopping experiences \cite{Hard2014applications}. In auto-mobile industry, virtual assistants are starting to integrate themselves with smart vehicles. As a fact, Microsoft already announced its {\em connected-vehicle} platform this year which tries to expand its Cortana capabilities on smart cars \cite{Baron2017assistants}. Nissan revealed in January this year that it planned to use Microsoft's vehicle platform in its cars. In addition, Apple has established its Car-Play system on Siri-enabled smart cars or via Siri on smart-phones. Furthermore, Hyundai is putting Google Assistant and Amazon Alexa into their cars to create some cool functions such as starting your car from the living room \cite{Baron2017assistants}.

Amy, Shae and Otto, these are three new virtual assistants on the rise in the market. Though they are only in either beta test or prototype stage, they represent the future of virtual assistants and what we can achieve \cite{Elgan2016future}.

Amy is created by a start-up company in New York called {\em x.ai}. In its current stage, Amy is only an email address: {\em amy@x.ai}. The goal for {\em x.ai} is to integrate Amy with other platforms such as Amazon Alexa, Slack and so on. Amy is invisible meaning there is nothing you need to install. For now, Amy is good at scheduling meetings. For example, when you want to schedule a meeting, simply cc the details to Amy's email address and your meeting will be scheduled. Amy can understand our daily language because it is powered by natural language processing techniques. For example, When you want to schedule a meeting with a colleague, you can send him an email and cc Amy, saying something like {\em Hi, how about we meet up some time}. Amy will then look into your calendar and come up with a time to meet. 

Amy can also interact with others. For example, when the person you plan to meet replies with certain conditions, Amy can take those conditions into account and suggest another meeting time all by herself without any human interventions. However, you can always ask Amy progress of the meeting scheduling by sending her an email. Last but not least, Amy can also reschedule meetings in the same way. The only thing you need to do is to send an email to Amy saying something like {\em Cancel and reschedule my meetings next week}. No matter how many meetings you have, Amy will contact all those people to reschedule the meetings and then update your calendar accordingly \cite{Elgan2016future}.

Shae is created by a company called {\em Personal Health 360} or {\em PH 360} to help people live a fresh and healthy life everyday by recommending health related suggestions and information. The company says that it uses big data as in more than ten thousand data points and several hundred mathematical algorithms to provide personalized health advice to users. The company claims these algorithms take into account factors such as medical history, family history, body type, demographics and even location data such as weather. These data are collected via surveys given by the company when users register for their accounts. If you have wearables like Fitbit, Shae can also obtain bio-metrics information from it to detect your stress level. In cases where it detects high level of stress, Shae will ask you whether you are experiencing any discomforts such as heart attacks. Like other health apps, Shae will give you advice on how to have a healthy diet and when to exercise as well as keeping an eye on your body measurements \cite{Elgan2016future}.

Otto is develop by Samsung to be a home virtual assistant like the Amazon Echo. Like the Echo, Otto is Internet-enabled and can interact with users through voice commands. It can also control smart devices in your home such as lights and thermostats. In addition, Otto can play music and help you in shopping online such as finding or placing an order. However, the difference between Otto and Echo is that Otto can also act as a security camera for your home when you are away. It can stream live videos to your personal smart-phone devices and you can also remotely control Otto's camera to look around the room \cite{Elgan2016future}.

All these three virtual assistants are unique in their own ways with Amy specializing in scheduling meetings, Shae in providing information on healthy lives and Otto in making homes safer. They demonstrate how smart, powerful and helpful virtual assistants will be in the future \cite{Elgan2016future}.

\section{Conclusion}

    Virtual assistants are incredibly helpful artificial intelligence machines which 
    utilize machine learning and speech recognition techniques to learn preferences 
    and behaviours to make our everyday lives better. We give a brief introduction 
    to virtual assistants and how they work. In addition, we list several current 
    virtual assistant applications in the industry created by giants like Google, 
    Apple, Microsoft and Amazon as well as how they differ in their specializations 
    and unique usages. We further discuss the future of virtual assistants with 
    emerging examples such as Amy, Shae and Otto to show how helpful virtual assistants 
    can be in the future.

\begin{acks}

  The author would like to thank Dr. Gregor von Laszewski for his
  support and suggestions to write this paper.

\end{acks}

\bibliographystyle{ACM-Reference-Format}
\bibliography{report} 