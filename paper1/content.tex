\title{Big Data Applications in the Media and Entertainment Industry}


\author{Jiaan Wang}
\affiliation{%
  \institution{Indiana University Bloomington}
  \streetaddress{3209 E 10th St}
  \city{Bloomington} 
  \state{Indiana} 
  \postcode{47408}
}
\email{jervwang@indiana.edu}


\begin{abstract}

    The growth of big data and its various applications in the media and entertainment 
    industry has been swift in recent years as well as the rapid surge of big data 
    and the increasing need for big data technologies. We describe the problems 
    that come with big data and its challenges in the industry. We then present 
    various utilization of big data and why big data is important to the advancement 
    of the media and entertainment industry. 
    
\end{abstract}

\keywords{i523, hid233, Big data, Media, Entertainment industry, Technology, Recommendation Systems}

\maketitle

\section{Introduction}

The amount of data being generated is increasing exponentially every year. Currently, we don't have the resources to process or analyze all the data. For example, giant tech companies like Google process over 20 petabytes of data daily \cite{Schlieski2012data}. The problem is that data are being generated very quickly and our analytics ability can't catch up with its pace. The goal is to turn these huge amount of data from a burden into decisions and knowledge \cite{Browning2015laptop}. Despite that, the technologies used to collect, analyze and interpret data are continuously improving \cite{Schlieski2012data}.

IDC, the International Data Corporation, believes that companies who take advantage of big data resources to help them make business decisions will flourish while those who don't fully utilize the power of big data will surely lose their competitive edge in the market and find themselves obsolete. This will be especially accurate for companies who face high rates of changes in business \cite{Villars2011care}.

But what exactly is big data? Wanda Group, a multinational conglomerate company based in China, defines big data as a DIKW hierarchical model, which stands for Data, Information, Knowledge and Wisdom \cite{Zhang2017era}. Big data is about the rising challenge that companies face as they handle vast and rapidly-increasing sources of data and knowledge that further introduce a complicated field of inquiries and use issues. Big data technologies define a new era of technologies and frameworks which aim to efficiently gather useful information from huge array of data by using various data science techniques \cite{Villars2011care}.

Emerging sources for big data include industries that are preparing to digitize their content. Particularly, the media and entertainment industry has just started content digitizing five years ago in areas such as ``digital recording, production and delivery'' \cite{Villars2011care}. Currently, the industry is gathering ``large amounts of rich content and user viewing behaviors'' \cite{Villars2011care}. This will essentially prepare them to adapt for the upcoming big data era and make good use of these so called big data. 

\section{Challenges in the Media and Entertainment Industry}

The issue with the immense data gathering and distributing structure we have created is: ``big data is a big mess'' \cite{Schlieski2012data}. All the information and data captured in our everyday lives simply have nowhere to be processed and it just gets put into storage forever \cite{Schlieski2012data}.

The media and entertainment industry has always been embracing new technologies because companies believe that big data technologies are crucial to solving their business problems such as reducing operating expenses in an increasingly competitive market and generating enough revenue from producing data and content from various platforms and products \cite{Lippell2016sectors}.

Traditional TV media is facing challenges as its data are scattered. It has physical data which comes from ``set-top boxes, network management systems, BOSS systems, etc.'' \cite{Zhang2017era} as well as online data which comes from user behaviors. The main challenge for traditional TV media in big data applications is data integration \cite{Zhang2017era}. In China, the overall economy of traditional TV media does not look promising. The amount of time user spent on traditional TV has declined while more time is spent on Internet TV. Studies have shown that, ``in 2012, Internet TV user base has reached 26.1 million while traditional TV user base is only 600 million'' \cite{Zhang2017era}. In addition, ``traditional TV operation rate has decreased from 70 percent in 2009 to 30 percent in 2012'' \cite{Zhang2017era}.

These are the main challenges the media and entertainment industry need to deal with in order to better utilize big data to make a difference:
\begin{itemize}

  \item Understand different big data sources, whether it is structured or unstructured, such as social media feed, emails, audio and so on. Insights and values can be gained from analyzing these sources to develop better products \cite{Lippell2016sectors}.
 
  \item Use complex mathematical algorithms along with domain expertise and information gathering platforms to select and organize information from vast amount of data \cite{Lippell2016sectors}.
 
  \item Businesses need to quickly get used to big data. Consumers nowadays are getting more familiar with the idea of big data. In addition, the cost of storage and analytics tools has been greatly reduced. Hence, it is extremely critical for businesses to understand the efficient use of big data to meet their needs and properly set up non-technical aspects such as management of personnel and staff in advance \cite{Lippell2016sectors}.
 
\end{itemize}

\section{Applications in the Media and Entertainment Industry}

Enormous amount of information is already being obtained about the entertainment industry \cite{Schlieski2012data}. For example, approximately 112 hours of footage has been captured from a horror TV series, {\em Supernatural} created in 2005 by Eric Kripke with its seventh season in progress. Contained in the footage is information about characters, their actions, dialogues and when, where and how each character dies. These are essentially data that we could use to create a map to the world of {\em Supernatural} and all its elements. However, currently, all these data are not being used and are stored somewhere, away from us \cite{Schlieski2012data}.

However, suppose we take all these information and store it in a system. Later, we apply analytics tools using our powerful and growing processing power. Then it is possible that we can create and interact with the world and the characters of {\em Supernatural} \cite{Schlieski2012data}.

Big data applications are also widely used in film industries as well. In Indian, a media company teamed up with IBM and ran their predictive modeling algorithm for the movie Ram Leela based on its social media buzz. With the proper selection of cities, the result produced was promising, with a 73 percent success for the movie. In addition, this predictive modeling analysis on social media data was also conducted for movies such as Barfi and Ek Tha Tiger, both of which achieved big success in the film industry. One of the most successful movies in 2013, Chennai Express by Shah Rukh Khan, also used big data analytics techniques supported by {\em Persistent Systems}, an IT service company, to boost up its social media buzz and marketing strategies. Tweets about Chennai Express generated ``over 1 billion cumulative impressions'' \cite{Karania2014industry} with ``more than 750 thousand'' \cite{Karania2014industry} related hash tags in total on twitter over the campaign period of 90 days. Crayon, a big data analytic firm that is based in Singapore, teamed up with producers from leading Hindi film industries to select and release the right kind of music for movies so that they create the perfect hype. Furthermore, Lady Gaga and her associates also used optimization techniques to create the best impact at her live events by going through listening preferences \cite{Karania2014industry}.

Another field where big data applications are making influence is sports. Germany, FIFA 2014 champion, has been utilizing SAP's Match Insights software to analyze team performance which made a big difference for the team. It analyzes data such as {\em touch maps} of player positions, passing abilities, ball retention and so on. In addition, Match Insights was also used by an Indian Premier League team, Kolkata Knight Riders to test the consistency of its players which helped in both auction and ongoing training as well \cite{Karania2014industry}.

In order for entertainment companies to create better products and advertising strategies to appeal to more clients, they should figure out why their customers subscribe or unsubscribe using big data analytics \cite{Mehta2017entertainment}. For example, using big data to analyze unstructured data sources such as social media feeds, emails and call records can often reveal reasons, often ignored, for stimulating customer interest. Furthermore, big data analytics also enables the possibility to create personalization systems by combing knowledge from the media and entertainment industry with basic user demographics \cite{Mehta2017entertainment}.

One of the most impressive and powerful personalization tools created is recommendation engine. It gathers information from people's past records and suggests or predicts items that they might like. Companies like Amazon has profited by successfully combining its recommendation engine with its online shopping experience from browsing goods to checkout \cite{Arora2016battle}.  

Artificial intelligence and deep learning technology are just two of the areas that Amazon is spending a vast amount of funds and resources in order to improve its recommendation engine to serve customers more effectively \cite{Arora2016battle}. In May 2016, Amazon announced its complex artificial intelligence platform called {\em DSSTNE}, pronounced as {\em destiny}. It is a cloud based open source AI framework created by Amazon to support and boost up its recommendation system \cite{Arora2016battle}.

In addition, Amazon claims that customers frequently watch and review pilots created and released at Amazon Studios. Executives from Amazon then choose which pilots will develop into a full series based on customer feedback. {\em Transparent}, a comedy show based on a transgender patriarch whose family lives in Los Angles, was one result from this system. After its initial release in 2014, it received positive reviews from the general public and raised awareness about transgender problems \cite{Whitley2016data}. It was rewarded the following year with ``the Golden Globe for best TV series, musical or comedy'' \cite{Whitley2016data}.

Another big media company who uses recommendation engines big time is Netflix who believes that content discovery is the number one priority. This is not surprising because its on-demand video and media streaming possibly dominates the world's market for digital content consumption. Just like Amazon, Netflix has spent immense amount of expenses and resources to make sure that its recommendation engine is one of the best to display its content library as much as possible \cite{Arora2016battle}. In December 2015, Netflix remodeled its recommendation system, ``deciding to do away with region based preferences'' \cite{Arora2016battle} due to their continue international expansion. 

Netflix utilizes sophisticated statistical equations to advertise series that customers might like, for example, {\em House of Cards} and {\em Orange Is the New Black}. Those equations usually contains predictor variables such as popularity of the series, customers' previously watched content or their demographics \cite{Whitley2016data}.

The personalization systems that Netflix has created for its customers has helped to lower the number of subscription cancellations, saving more than 1 billion dollar a year \cite{Whitley2016data}.

\section{Conclusion}

The rapid growth of big data has given the media and entertainment industry an unique opportunity to utilize resources in order to benefit from big data applications and technologies. However, there are still some key challenges media companies are facing such as how to quickly adapt to the big data era, how to deal with and analyze immense amount of data pouring in every minute and how to make cost-effective products and consumer experiences. Examples of current effective big data applications and technologies such as Match Insights from SAP and personalized recommendation engines from Amazon and Netflix are provided. In summary, big data applications and technologies are crucial in the success of media and entertainment companies.

\begin{acks}

  The author would like to thank Dr. Gregor von Laszewski for his support and suggestions to write this paper.

\end{acks}

\bibliographystyle{ACM-Reference-Format}
\bibliography{report} 
