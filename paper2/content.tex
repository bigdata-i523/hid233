\title{Big Data Applications in Virtual Assistants}

\author{Jiaan Wang}
\affiliation{%
  \institution{Indiana University Bloomington}
  \streetaddress{3209 E 10 St}
  \city{Bloomington} 
  \state{IN} 
  \postcode{47408}
}
\email{jervwang@indiana.edu}

\begin{abstract}

    In the age of big data, artificial intelligence and speech recognition 
    techniques have been widely used in numerous big data technologies and 
    applications. Among those are virtual assistants which could potentially 
    lead to the future evolution of big data. We list various virtual assistants 
    currently in the industry developed by giants such as Google, Microsoft, 
    Amazon and Apple. We then follow up by discussing some future development 
    of virtual assistants.
    
\end{abstract}

\keywords{i523, HID233, Big data, Virtual Assistants, Artificial intelligence}

\maketitle

\section{Introduction}

Since the early 2000s, big data has been a popular word among the tech industries. It is a term that describes data sets which are so huge that normal data management techniques are not enough to process them. Nowadays, big data usually refers to data science analytics methods such as predictive modeling, machine learning, data mining, speech recognition and so on \cite{Tal2015internet}. Big data is typically described by the 4Vs, that is Volume, Velocity, Variety and veracity. Volume stands for the size of the data, velocity refers to the data processing speed, variety means different types of data and finally veracity represents the quality of the data. 

As the name suggests, big data is big. The size of the data could go up to petabytes or even exabytes which is 1000 petabytes. Previously, to process this amount of data is essentially impossible. However, as the year goes by, the ability to analyze big data has increased significantly thanks to improved mathematical algorithms and powerful computers \cite{Tal2015internet}. 

In the era of big data, the opportunities are immense. Big data not only will help all kinds of companies and organizations to make better services and decisions but also will aid us in everyday tasks such as writing emails, sending texts, shopping online, listening to music, watching TV, traveling around the globe, managing schedules, and so on \cite{Tal2015internet}. This led to to the birth of virtual assistants which are software that can perform various tasks described above for individuals. The major technologies behind virtual assistants are artificial intelligence and speech recognition which would not be possible without the use of big data \cite{Baron2017assistants}. At present, more and more companies and organizations are beginning to employ the use of virtual assistants which utilize artificial intelligence. As a result, along with the help of machine learning algorithms, more and more applications, technologies and systems will be able to obtain information on their own such as user behaviors to predict and provide useful suggestions to individuals with more personalized experiences \cite{Hard2014applications}.

Virtual assistants work usually via three different methods: by texts, or voices or sometimes images. A virtual assistant may work via more than one method such as Goggle Assistant which can understand texts and voices. After users input their commands either via texts or voices, the virtual assistants process those information received using a method called natural language processing which then translates user inputs into executable commands they can interpret. 

\section{Current applications in the industry}

In this new era of virtual assistants, four major tech companies come into play: Google with Google assistant and Google Now, Apple with Siri, Microsoft with Cortana and Amazon with Alexa. All these virtual assistants have been successful in the market but the technologies behind them are still immature. The four giants have been competing with each other in this area for years and each of them wants to bring their own unique virtual assistant to the market. 



``Take Google for example. Creating a single Google account gives you access to its large ecosystem of free services - search, email, storage, maps, images, calendar, music and more - that are accessible from any web-enabled device. Every action you take on these services (thousands per day) is recorded and stored in a {\em personal cloud} inside Google's server farms. With enough use, Google begins to understand your preferences and habits with the end goal of using {\em anticipatory systems} to provide you with information and services you need, when you need it, before you even think to ask for it'' \cite{Tal2015internet}.

``Google already has a sizable advantage in the machine learning field. Because of their globally dominant search engine, popular ecosystem of cloud-based services like Chrome, Gmail, and Google Docs, and Android, Google has access to over 1.5 billion smart-phone users. This is why heavy Google and Android users will likely choose a future version of Google's VA system, Google Now, to power their lives'' \cite{Tal2015internet}.

``Apple has also made considerable headway in overcoming the early disappointment by adding more capabilities to Siri and helping users better understand what it can and cannot do'' \cite{Waters2015life}. ``For example, Apple users generally use Apple desktops or laptops at home and Apple phones outdoors, all while using Apple apps and software in between. With all these Apple devices and software connected and working together within the Apple ecosystem, it shouldn't come as a surprise that Apple users will likely end up using Apple's VA: A future, beefed up version of Siri. Non-Apple users, however, will see more competition for their business'' \cite{Tal2015internet}.

``While seen as an underdog due to its near non-existent market share in the smart-phone market, Microsoft's operating system, Windows, is still the dominant operating system among personal desktops and laptops. With its 2015 roll-out of Windows 10, billions of Windows users around the world will be introduced to Microsoft's VA, Cortana. Active Windows users will then have an incentive to download Cortana into their iOS or Android phones to ensure everything they do within the Windows ecosystem gets shared with their smart-phones on the go'' \cite{Tal2015internet}.

``The value of a virtual assistant is having it there wherever you are, giving you the tailored information you need sometimes before you even know you need it. Microsoft, more than the other tech giants building out this tech, has deeper roots in business software and productivity. Cortana is enabled across a number of Microsoft's apps and services-from Microsoft Power BI to Skype to provide immediate contextual responses to business queries without leaving the app you are in'' \cite{Marvin2017them}.

``The simplest way Cortana does this is through reminders. Scheduling, reminders, and lists are a top-of-mind business use case for virtual assistants. Jones talked about using Cortana in a touchscreen device such as the Microsoft Surface Pro. Smart Sticky Notes in the Windows 10 Anniversary Update let you write something like, {\em Call my boss at 3pm}, either by typing or by writing a note with the stylus as part of Windows Ink. Cortana will then add that reminder to keep track of the task'' \cite{Marvin2017them}.

``Microsoft is also working with Wunderlist to integrate Cortana and sync lists across devices. This is all part of a more proactive strategy, using both contextual data and location-based reminders to help users manage their emails, schedule, and day-to-day commitments. Microsoft is looking to expand this even further to dynamically create Cortana to-do lists and surface information based on data throughout Office 365. Cortana is already fully integrated into the Microsoft Edge browser and can search for documents or people across apps such as and SharePoint'' \cite{Marvin2017them}.

``The Windows 10 Creators Update also integrated Cortana with Microsoft Azure Active Directory (AAD) to bring the AI capabilities to enterprise users who may not have had access to it before. These kinds of integration also extend to Power BI, which lets you pull Cortana data into business intelligence queries and reports. That's not to be confused with the Cortana Intelligence Suite, a separate enterprise offering that builds machine learning and predictive analytics into business apps'' \cite{Marvin2017them}.

``Beyond that, Jones said the Cortana team is working with Microsoft Research through projects such as {\em Calendar.help} to automate processes like scheduling meetings with contacts outside your organization. The team is also working with the Microsoft IT Division development team to create experiences specific to Cortana that pull in a range of apps and contextual data'' \cite{Marvin2017them}.

``Cortana and Google Assistant are getting smarter, with contextual reminders and recommendations geared toward optimizing productivity along with fascinating innovations incorporating computer vision and other machine learning algorithms. Alexa is building out a diverse ecosystem of third-party skills, and Google and Microsoft have followed suit'' \cite{Marvin2017them}.

``Alexa Skills Kit, Cortana Skills, and Actions on Google give companies and developers the tools to apply the voice tech to everything from email marketing and e-commerce to expense tracking and fleet management. These business applications and use cases are only what we've seen so far. PCMag spoke to execs from Amazon, Google, and Microsoft to understand their virtual assistant vision, how the tech is evolving, and what these companies believe businesses can do with voice-capable AI helpers'' \cite{Marvin2017them}.

``Amazon is the standard-bearer in this regard. The Alexa Skills Kit has been available since 2015 and lets companies and developers apply Alexa to whatever business environment or process they desire. As a result, there is already a wide selection of available business skills that companies can simply enable and start using - and that ecosystem is growing'' \cite{Marvin2017them}.

``Google and Microsoft have followed Amazon's lead on this front with Actions on Google and Cortana Skills, respectively. These tool-kits let you build particular skills but they're also evolving to incorporate natural language processing and features such as proactive suggestions to recommend a skill to users in the right context'' \cite{Marvin2017them}.

``Now, perhaps unsurprisingly, Facebook pops up again. In the last chapter for this series, we mentioned how Facebook will likely enter the search engine market, competing against Google’s fact-focused semantic search engine with a sentiment-focused semantic search engine. Well, in the field of VAs, Facebook can also make a big splash'' \cite{Tal2015internet}.

``Facebook knows more about your friends and your relationships with them than Google, Apple, and Microsoft together ever will. Initially built to compliment your primary Google, Apple, or Microsoft VA, Facebook’s VA will tap into your social network graph to help you manage and even improve your social life. It’ll do this by encouraging and scheduling more frequent and engaging virtual and face-to-face interactions with your friend network'' \cite{Tal2015internet}.

``Over time, it is not hard to imagine Facebook's VA knowing enough about your personality and social habits to even join your circle of true friends as a distinct virtual person, one with its own personality and interests that reflect your own'' \cite{Tal2015internet}.

``Today, the aid these virtual assistants provide remains limited. Most users of Google Home and Amazon Echo devices - which host Assistant and Alexa, respectively - stream music, play audio books, and control smart-home devices, according to surveys by San Francisco analytics firm VoiceLabs'' \cite{Baron2017assistants}.

``Still, the virtual agent's foundation in AI means the more it learns about a user's preferences and behaviors, the better job it can do. So while experts predict a handful of firms will dominate in this field - most agree Apple, Google and Amazon will be major players, with Microsoft in a lesser role - they are split on whether consumers will be served best by one bot, or more'' \cite{Baron2017assistants}.

``People want one assistant, they do not want two. You want one assistant, to be very readily available wherever you are. However, the various assistants will likely end up somewhat specialized in their expertise, with Google Assistant, for example, excelling in providing knowledge and managing schedules, and Microsoft Cortana leading on gaming. In a few years, many people will use two or three different assistants'' \cite{Baron2017assistants}.

``For all the major players, virtual assistants provide important data that fuels the AI that powers and improves them, making both the assistants and the products they live in ever more marketable. For Amazon, Alexa is an enthusiastic purchasing agent for the e-commerce that drives the firm. For Google, Assistant is a turbocharged vacuum for the data the company collects to sell ads targeted directly at users \cite{Baron2017assistants}.

\section{The future of virtual assistants}

``The ideal virtual assistant of the future will help fill in the gaps that currently exist between a personalized experience that in-store shoppers are used to, and current, less helpful, online and mobile shopping applications. Some virtual assistants have already begun to use geo-targeting technologies to localize experience, but the next generation will focus on a direct interaction with the customer that will create a seamless customer experience – handling every step of the purchase cycle. We will continue to see more of these emerge as omni-channel retail becomes a larger part of our everyday lives, beginning with mobile virtual assistant apps'' \cite{Hard2014applications}.

``Putting these robo-helpers into cars' on-board systems has become a priority for major firms, including Microsoft, which seeks to extend the reach of its PC-based Cortana through the {\em connected-vehicle} platform it announced this year'' \cite{Baron2017assistants}.

``In January, Nissan announced it would integrate Microsoft's platform into its cars. Siri already can be used in a car via a phone or Apple's CarPlay system, or in cars sold with Siri integration built in. Hyundai is bringing Alexa and Google Assistant into some of its cars so, for example, an owner could start their car from their living room'' \cite{Baron2017assistants}.

``A new crop of virtual assistants is on the way, led by Amy, Shae and Otto. Each in its own way represents the future of virtual assistants. One is in public beta, one is in private beta and one is a hardware prototype, but they are coming soon, and they collectively reveal how much better virtual assistants can be'' \cite{Elgan2016future}.

``Amy does one thing really well: scheduling your meetings. Amy is the creation of a New York startup called {\em x.ai}. For now, Amy lives on the other side of an email address: {\em amy@x.ai}. Company intends to put Amy on other platforms, such as Slack and other group-chat apps, Amazon Echo and more, and that the platform shouldn't matter'' \cite{Elgan2016future}.

``You simply cc: Amy's email address on your communication about the scheduling of any meeting, and Amy takes over. Amy is {\em invisible software} - there is no app to install, no website to interact with'' \cite{Elgan2016future}.

``Amy is adept with natural language processing, which means you can use everyday language. For example, you might send an email to a colleague, copying Amy, and say: {\em Hey, let's get together next week} or {\em Grab a bite next week}? or {\em We should connect}. Amy will then take action, introduce herself to the other person and, based on your calendar and preferences, suggest a time to meet'' \cite{Elgan2016future}.

``Amy is interactive. If the person you want to meet with gets back to Amy with restrictions or additional suggestions, Amy handles all the back-and-forth that often attends the hunt for a mutually agreeable meeting time. If you want to know how it's all going, you can send an email to Amy and ask how the meeting with so-and-so is going and Amy will reply with the current status. If anyone wants to reschedule later, Amy handles that in the same way'' \cite{Elgan2016future}.

``Best of all, Amy does the heavy lifting when you need to reschedule. Let's say you decide to take a last-minute vacation. Just send an email to Amy and say: {\em Clear my schedule for next week}. If you have got 10 meetings scheduled, Amy will reach out to all 10 people to reschedule and will update your calendar'' \cite{Elgan2016future}.

``Amy represents the future of virtual assistants for two reasons. First, it's a specialist agent, doing one thing very well. Second, Amy is believably human. Within the confines of email conversations on the subject of scheduling meetings, Amy passes the Turing test'' \cite{Elgan2016future}.

``Shae helps you get healthy by guiding and informing you about healthy living all day, every day. The company behind Shae, {\em Personal Health 360} or {\em PH 360} throws around some big numbers. It claims Shae uses some 500 algorithms fed by more than 10,000 data points to provide very specific help for users'' \cite{Elgan2016future}.

``That is a lot of data, and it comes from unexpected places. For example, family history is taken into account, individual body type and environmental factors like the weather and pollen count. Much of the health data Shae uses comes from a personalized phenotype questionnaire that each user fills out'' \cite{Elgan2016future}.

``Shae additionally accesses both your calendar and bio-metrics as detected from a monitoring device like the Apple Watch to figure out what your mood might be. When it detects signs of stress such as an elevated heart rate, the app pops up a dialog to ask you if you are feeling stressed'' \cite{Elgan2016future}.

``Like Google Now, Shae takes the initiative to give you information, updates and advice, telling you what to eat and when to exercise, and keeping tabs on changing health data, such as your weight, body mass index, lean muscle mass and other body measurements. Shae even helps you plan vacations based on your personal profile and circadian rhythm'' \cite{Elgan2016future}.

``Over time, we will learn if users are thrilled with the Shae assistant. Whether Shae succeeds or fails, the service represents the future of virtual assistants because of its extreme personalization and the eclectic nature of the data, integrating family history, personal health details, health knowledge, environmental data and more - and for its preemptive advocacy of habits that benefit the user'' \cite{Elgan2016future}.

``Following the success of Amazon's Echo device, Samsung unveiled its own virtual assistant home appliance called Otto in April of this year. Like the Amazon Echo, which is possessed by a virtual assistant named Alexa, the Otto is an Internet-connected speaker and microphone that interacts with you via spoken conversations and can control home appliances like lights. Otto can answer questions, order products, and play music and pod-casts on command. But unlike the Echo, Otto is also an HD security camera that can stream video live to your phone or computer. The device has a kind of head that can turn, pivot and swivel to let you look around the room remotely. Otto can also recognize faces - and even has a rudimentary face of its own, displayed on a screen'' \cite{Elgan2016future}.

``It is based on Samsung's ARTIK IoT platform, which Samsung recently unveiled developer tools for. Otto represents the future of virtual assistants not only because it is a physical home appliance, but also because it uses facial recognition. That means different members of the family could each have their own set of preferences and personal details, calendars and accounts - and Otto and other future appliances could base their interactions on the knowledge of the person they are talking to'' \cite{Elgan2016future}.

``Shae, Amy and Otto together represent the future of virtual assistants, which will be specialized, personalized, thorough, preemptive, highly intelligent and optionally available in the form of dedicated physical appliances. These three virtual assistants already suggest just how helpful and, well, human our technology will become'' \cite{Elgan2016future}.

\section{Conclusion}

Put here an conclusion. 

\begin{acks}

  The author would like to thank Dr. Gregor von Laszewski for his
  support and suggestions to write this paper.

\end{acks}

\bibliographystyle{ACM-Reference-Format}
\bibliography{report} 