\title{Big Data in Safe Driver Prediction}

\author{Jiaan Wang}
\affiliation{%
  \institution{Indiana University Bloomington}
  \streetaddress{3209 E 10 St}
  \city{Bloomington} 
  \state{IN} 
  \postcode{47408}
}
\email{jervwang@indiana.edu}

\author{Dhawal Chaturvedi}
\affiliation{%
  \institution{Indiana University Bloomington}
  \streetaddress{2679 E 7th St}
  \city{Bloomington} 
  \state{Indiana} 
  \postcode{47408}
}
\email{dhchat@iu.edu}

\begin{abstract}

    For years, people have been trying to reduce their automobile
    insurance bills. Insurance companies claim that price will be
    reduced for good drivers and raised for bad ones. However,
    inaccuracies in their data predictions lead to the exact
    opposite. The dataset being used is released by Porto Seguro,
    an auto and homeowner insurance company from Brazil. It
    consists of information from several hundred thousands of
    policyholders. The goal is to predict the probability an auto
    insurance policyholder files a claim the next year using
    classification algorithms. A good prediction with decent
    accuracy can correctly adjust prices for policyholders.
    
\end{abstract}

\keywords{i523, HID233, Big data, Classification, Safe Driving, Predictive Analytics, Neural Networks}

\maketitle

\section{Introduction}

``Predictive analytics can be a powerful decision-making tool within various parts of fleet operations. It’s used in many major industries such as retail, finance and insurance to forecast what is most likely going to happen in the future — it’s a popular risk assessment tool. To illustrate the potential uses in the world of fleet, safety and safety-related decisions provide a great example of how predictive analytics can impact fleet operations'' \cite{Suizo2015decisions}.

``Fatal vehicle collisions are one of the leading causes of death in the U.S., according to the Centers for Disease Control. While this may not come as much of a surprise considering these types of fatalities frequently make headlines, what should be more alarming is the fact that many of these collisions are preventable'' \cite{Suizo2015decisions}.

``With the nature of fleet business often revolving around its employees being on the road, fleets are significantly impacted by these statistics. Automotive Fleet reports most company drivers average 20,000 miles per year, with more fleets experiencing an increase in preventable accidents. The primary cause of this uptick in preventable fleet accidents points to driver distraction that now contributes to the 25-30 percent of all fleet-related accidents, reports Automotive Fleet'' \cite{Suizo2015decisions}.

``With over 40,000 deaths each year coming from traffic-related collisions and accidents, it's a clear sign that improving road safety is a top priority across the nation. Advances in technology are helping reduce accidents and improving overall driver safety through a variety of methods. Here are the top 4 ways that technology will help improve road safety in the coming years'' \cite{Mills2017safety}.

``For years, insurance companies have used estimates of your annual mileage to determine your car insurance rates. But with recent changes in technology, insurers now have an unprecedented ability to judge your actual driving habits. Armed with detailed data on how often you slam on the brakes and what times of day you're on the road, insurance companies are increasingly relying on precise, technological means of assessing risk — and using that information to set your monthly premiums'' \cite{Fung2016turn}.

``Along with computer controlled vehicles, data collection is vital to ensuring that we understand where and why accidents happen. The black box technology that has been famously used to track airplanes and help identify the cause of crashes is now being used in other vehicles. Black box technology is fairly simple, inexpensive, and easy to deploy on a wide scale of cars. The benefits of the technology are we will be able to track the exact time, speeds, position, and other factors related to car collisions and accidents. As this data is studied, we will be able to better understand trends and reasons behind car crashes and use this data to prevent future incidents. South Korea was the first country to deploy black box technology in their taxi services and immediately noticed a 14 percent decrease in traffic accidents the following year. The death toll and injury rate associated with traffic accidents also decreased by as much as 20 percent'' \cite{Mills2017safety}.

``Through telematics and other data, predictive models make it possible for fleets to make more educated decisions with less expense — it provides support for decisions, making them more efficient and effective, or in some cases, can be used to automate an entire decision-making process. Considered the most revolutionary technological step of the “Big Data” era, predictive analytics has quickly become one of the most advanced forms of customized risk management'' \cite{Suizo2015decisions}.

``Predictive analytics is ideal for risky driving analysis, for example. The challenge with identifying high-risk drivers is the most transparent records such as driving records, traffic violations, and accident reports, and may not always indicate a driver with the highest potential risk. Armed with information on driving behaviors through telematics data, companies can put drivers in safety training programs tailored to their risky driving behaviors before a collision occurs. Information about a driver’s behavior can also be used to determine how likely an individual is to be involved in an accident as well as the costs associated with that risk'' \cite{Suizo2015decisions}.

``Outside of safety, by leveraging and analyzing data to provide insights into vehicle and equipment usage, driver behavior, and fleet productivity schedules, fleet managers can discover various areas where cost efficiency can be applied. This includes preventive maintenance, which can greatly contribute to cost cutting and increased productivity. Trending out maintenance data creates the opportunity to predict component failure, and provide real-time transaction repair costs, helping fleets to shape best-in-class maintenance policies and procedures'' \cite{Suizo2015decisions}.

``Looking at information such as past usage, maintenance and overall total cost of ownership, fleets can also use predictive analytics for vehicle procurement strategies. Predictive analytics can help in other ancillary ways as well. One fleet that implemented the driver safety score model in its operations noted an increase in personal communication, with drivers sharing information and essentially comparing notes and scores. After using the program for several months, the client reported drivers calling in to discuss safety proactively and in a positive way, whereas in the past, safety had only been a source of punishment for a driver when they made a mistake, so it was often a topic that drivers avoided. This truly changed how safety was viewed by the driver population and the organization as a whole'' \cite{Suizo2015decisions}.

``Predictive models come in various forms, depending on the behavior or event they are predicting. Most generate a score, similar to a credit score, with a higher score indicating a higher likelihood of the given behavior or event occurring. Predictive analytics encompasses a variety of techniques from statistics, modeling, machine learning and data mining that analyze current and historical facts to make predictions about future or otherwise unknown events'' \cite{Suizo2015decisions}.

``Predictive analytics tells the customers, with as much accuracy as possible, what will happen in a few weeks or other chosen time frame — not what has happened in the past, but what the most likely outcome for a specific driver or vehicle is in the future and why that’s the case. Predictive analytics gives fleets an opportunity to think proactively rather than reactively'' \cite{Suizo2015decisions}.

``Using safety as an example, by analyzing a driver’s particular driving style via telematics data, a safety report can be compiled highlighting the risks inherent to the driver and help coach safer driving habits. Drivers are classified into categories of risk based off the probability that a driver will be in a collision. By making these probabilities easy to assess, a safety score can help fleets make better decisions on how to prioritize risk within their operations. The primary objectives of a safety score would be to identify risky drivers prior to a collision to provide the driver an opportunity to modify those behaviors and prevent the collision from occurring'' \cite{Suizo2015decisions}.

``Like other types of risk assessment programs, predictive analytics is not intended to be a magic bullet. It is only the beginning and requires a total commitment from the organization, fleet personnel, and driver to turn the model’s efficacy into reality. It requires strong management commitment in order to succeed. The most successful users are organizations that involve all departments in the process at an early stage including operations, safety, HR and IT. One reason previous predictive analytics projects have failed is due to the failure to achieve alignment and full enterprise adoption'' \cite{Suizo2015decisions}.

``The main component in leveraging fleet analytics is assembling and analyzing actual fleet data. In order to develop accurate and valuable predictive models, it’s important to understand the challenges you want to address first to ensure that the models are solving a real-world problem. Real data and real problems are the key'' \cite{Suizo2015decisions}.

``Start by choosing the area of greatest need (e.g., safety, maintenance, or workers compensation). Large volumes of data can result in hours of analysis with no real return in the way of fleet operational and cost optimization. One step to curb data overload is to simply ensure the data accumulated is the data actually needed. Data and analytical models should align with overall fleet goals and provide measurable and meaningful results. One good place to start leveraging analytical data is to understand your organization’s goals beyond fleet as well. Being able to meet these goals through the use of measurable analytics is the objective'' \cite{Suizo2015decisions}.

``Next, identify the key elements and the type of data that will make up your fleet’s predictive model. The necessary data is then extracted, which should be validated by the fleet. Analytics involves more than just having the data — fleets must ensure they understand what the data is telling them and why it will help them predict a future event'' \cite{Suizo2015decisions}.

``The best way to balance the accuracy and timeliness of the predictive model, such as the driver safety score described earlier, is to run the models every few weeks or the time frame that makes the most sense for your operations. Giving it at least a few weeks provides safety and operational groups enough time to take action, while still accurately reflecting the changes that indicate risk. This is balanced with trying to minimize operational burden since most have limited time, whether it’s coaching drivers or needing to get a vehicle in for repair before a major breakdown'' \cite{Suizo2015decisions}.

``To further the safety example, the program’s success could be measured by the percentage of drivers involved in collisions that were predicted accurately in the 90-day period prior to the collision. More specifically, it would examine the data of the risky driver to determine if that individual was classified at the highest risk level at least one-third of the time in the 90 days prior. This example uses an odds ratio, which is the measure of association between an exposure and an outcome. To explain further, it looks at the odds a driver will get into a collision in the next 90 days given a particular score'' \cite{Suizo2015decisions}.

``Although predictive analytics is most commonly used with fleets who were early adopters of telematics, it is something that more and more fleets are seeing the value in. With predictive analytics, fleets can proactively take action and make educated decisions that positively impact their organizations by reducing risk, whether tied to improving safety and costs or decreasing the work behind the decision-making process'' \cite{Suizo2015decisions}.

\section{Current Applications}

``Predictive analytics are being employed in interesting new ways to improve safety. Telematics solutions have long monitored events like hard braking and speeding to flag unsafe driver behaviors.  But today, this driver event data is being enriched with other data streams to actually predict the likelihood of a specific driver having an accident. A company called SmartDrive Systems goes even further in their efforts to predict accidents.  They are enriching the telematics data with video feeds from road facing and interior facing cameras.  The company is combining asset sensor with driver sensor data to do better driver safety analytics and ultimately make better predictions'' \cite{Banker2016accident}.

``SmartDrive Systems is a predictive analytics supplier whose solution is based on a private cloud architecture.  In other words, all of their customers’ data is captured by the company; they have telematics and video on four billion miles driven, of which they scored almost 200 million events.   SmartDrive’s customer agreements allow them to analyze all the customer data so they can continue to improve their algorithms.  “We are constantly tuning and improving our algorithms,” Mr. Mitgang said. The combination of telematics and video has allowed their scientists to better interpret the telematics data. By reading the telematics data, and seeing what happened, they were able to determine, for example, that turning more than 165 degrees within a certain turning radius and time window was a risky U-turn on a roadway. SmartDrive also detected additional patterns to avoid false trigger activation in large open areas such as parking lots and truck stops'' \cite{Banker2016accident}.

``Indiana State Police decided to take a different approach, and are making their predictive crash analytics program available to the public, as well as troopers. A color-coded Daily Crash Prediction map, which went online in November, pulls together data that includes crash reports from every police agency in the state dating to 2004, daily traffic volume, historical weather information and the dates of major holidays, said First Sgt. Rob Simpson. The online map pinpoints where a crash is likely, ranging from a very low to a high probability. It also highlights prior crash sites and displays information about the date and cause, whether EMS was called, if there was a fatality, and if drugs or alcohol were involved. As of late January, the site had nearly 4,800 page visits'' \cite{Bergal2017sites}.

``Liberty Mutual, the country's third-largest property-and-casualty insurer, took the latest step in that direction Monday when it announced a partnership with Subaru. Beginning later this year, Subaru drivers who have paid for the automaker's Starlink infotainment system will be able to download an app to their cars that notifies them when they are accelerating too aggressively or braking too hard. The app is part of Liberty Mutual's RightTrack program, which gives drivers a 5 percent discount on their rates for enrolling and additional discounts up to 30 percent for heeding the app's guidance on driving safely'' \cite{Fung2016turn}. 

\section{Neural Networks}



\section{Analysis}



\section{Discussion}



\section{Conclusion}


\begin{acks}

  The authors would like to thank Dr. Gregor von Laszewski for his
  support and suggestions to write this paper.

\end{acks}

\bibliographystyle{ACM-Reference-Format}
\bibliography{report} 
